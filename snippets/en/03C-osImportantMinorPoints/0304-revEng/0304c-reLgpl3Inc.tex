% Telekom osCompendium 'for being included' snippet template
%
% (c) Karsten Reincke, Deutsche Telekom AG, Darmstadt 2011
%
% This LaTeX-File is licensed under the Creative Commons Attribution-ShareAlike
% 3.0 Germany License (http://creativecommons.org/licenses/by-sa/3.0/de/): Feel
% free 'to share (to copy, distribute and transmit)' or 'to remix (to adapt)'
% it, if you '... distribute the resulting work under the same or similar
% license to this one' and if you respect how 'you must attribute the work in
% the manner specified by the author ...':
%
% In an internet based reuse please link the reused parts to www.telekom.com and
% mention the original authors and Deutsche Telekom AG in a suitable manner. In
% a paper-like reuse please insert a short hint to www.telekom.com and to the
% original authors and Deutsche Telekom AG into your preface. For normal
% quotations please use the scientific standard to cite.
%
% [ Framework derived from 'mind your Scholar Research Framework' 
%   mycsrf (c) K. Reincke 2012 CC BY 3.0  http://mycsrf.fodina.de/ ]
%

Based on our experiences how to successfully carve out the meaning of license
text, we can shorten the way to understand the one LGPL3-RevEng-Sentence
referring to \emph{reverse engineering}:

\begin{quote}\emph{\enquote{You may convey a Combined Work under terms of your
choice that, taken together, effectively do not restrict modification of the
portions of the Library contained in the Combined Work and reverse engineering
for debugging such modifications, if you also do each of the following
[\ldots]}\footnote{\cite[cf.][\nopage wp., §4]{Lgpl30OsiLicense2007a}. The
ellipsis at the end of the sentence denotes a set of tasks which we do not
listen here for saving resources, but which have to be considered as an
integrated part of this sentence.}}
\end{quote}

Reusing our method of disambiguation, we first can exemplify the meaning of the
LGPL3-RevEng-Sentence by the following text:

\begin{alltt}   
( \([\Theta:]\)
  ( You \(\emph{compliantly distribute}\) a Combined Work 
    under terms of your choice 
    (   (that together effectively, do not restrict modification of 
        the portions of the Library contained in the Combined Work)
    \(\textbf{AND}\) 
        (that together effectively, do not restrict reverse
        engineering for debugging modifications of the portions
        of the Library contained in the Combined Work)
  )  )
  \(\textbf{IF}\)
  \([\Omega:]\) 
  ( you also do each of the following \([\ldots]\))
)
\end{alltt}

But now, a simply executed logical serialization let us running into a problem:

If we serialized \emph{($\Theta$ IF $\Omega$)} as \emph{($\Omega$ $\rightarrow$
$\Theta$)}, then from not respecting $\Theta$ would follow by Modus Tollens,
that we are not allowed to realize $\Omega$ -- in other words:
that we may not do even one of the single tasks covered by the ellipsis -- which
is a silly result. 

If we serialized \emph{($\Theta$ IF $\Omega$)} as \emph{($\Theta$ $\rightarrow$
$\Omega$)} then from doing $\Theta$ would successfully follow by Modus Ponens
that we also have to do $\Omega$. And from not respecting $\Omega$ would
successfully follow by Modus Tollens, that we must not do $\Theta$. But
unfortunately, we can respect this second interdiction also \emph{by
distributing a Combined Work under terms} that restrict modifications and/or
reverse engineering (instead of not restricting these techniques) -- which,
again, is a silly result.

Obviously, a simple serialization based on a intutively unclear reading fails.
In fact, the LGPL3-RevEng-Sentence must have a more sophisticated underlying
structure. It must be logically serialized in a form, that integrates the
requirements, not to restrict modifications and reverse enigneering, as really
triggable conditions. Thus, the meaning of the sentence can logically be
explicated as the \emph{LGPL3-RevEng-Rule}:

\begin{alltt}
( \([\Sigma:]\)
  ( You \(\emph{compliantly distribute}\) a Combined Work 
    under terms of your choice 
  ) 
  \(\rightarrow\)  
  (    \([\Gamma:]\)
     ( the terms of your choice together effectively do 
       not restrict modification of the portions of the 
       Library contained in the Combined Work) 
     \(\wedge\) \([\Delta:]\)
     ( the terms of your choice together effectively, do 
       not restrict reverse engineering for debugging 
       modifications of the portions of the Library 
       contained in the Combined Work)
     \(\wedge\) \([\Omega:]\) 
     ( you also do each of the following \([\ldots]\))
) )
\end{alltt}  

This LGPL3-RevEng-Rule indeed successfully regulates how to compliantly
distribute a Combined Work by telling us,

\begin{itemize}
  \item that we have to respect $\Gamma$, $\Delta$ \textbf{and} all single parts
  of $\Omega$, if we distribute a Combined Work compliantly\footnote{follows by
  Modus Ponens. Thus, in this case especially our terms \enquote{[\ldots]
  together effectively \textbf{[must] not restrict reverse engineering} for
  debugging modifications of the portions of the Library contained in the
  Combined Work}.}.
  \item that we do not distribute a Combined Work compliantly, if we do not
  respect one of the requirements $\Gamma$, $\Delta$ or one of the single parts
  of $\Omega$\footnote{follows by Modus Tollens. Thus, especially we are not
  distributing a Combined Work compliantly, if our terms \enquote{[\ldots]
  together effectively \textbf{do restrict reverse engineering} for debugging
  modifications of the portions of the Library contained in the Combined
  Work}.}.
\end{itemize}

Now, we can directly see, that the LGPLv3 does not enforce us, not to obstruct
reverse engineering in all respects! The required reverse engineering is limited
to the purpose of supporting the debugging of modifications and focused to the
Combined Work containing portions of the Library. In other words: our terms may
obstruct other purposes of reverse engineering or may restrict reverse
engineering of other forms of our work which which can not be specified as
Combined Work or do not contain portions of the Library. Thus, the first crucial
question is, what the LGPL-v3 means if it talks about a \enquote{Combined Work}.
The second question is, what the LGPL-v3 specifies as a portion of the Library.

Again, fortunately, the LGPL-v3 answers clearly: \enquote{A \enquote{Combined
Work} is a work produced by combining or linking an Application with the
Library}\footcite[cf.][\nopage wp., §0]{Lgpl30OsiLicense2007a}. From our LGPL-v2
analysis we know the ways how works that uses a Library can technically be
linked or combined with the Library:

\begin{itemize}
  \item Copying code from the Library into the work using the
  Library\footnote{The LGPL-v3 designates the work using the Library as
  \enquote{Application} and defines that it \enquote{[\ldots] makes use of an
  interface provided by the Library [\ldots]} (\cite[cf.][\nopage wp.,
  §0]{Lgpl30OsiLicense2007a}).} causes that the application respectively the
  work using the Library indeed contains portions of the
  Library\footnote{$\rightarrow$ p. \pageref{RevEngCopyCodeManually}}.
  \item Combining script language based applications and Libraries may evoke
  that the resulting application contains portions of the Library. But the
  details can be neglected with respect to the reverse engineering, because
  script code is distributed as it has been developed and can therefore directly
  be understood\footnote{$\rightarrow$ p. \pageref{RevEngDistributeScripts}}.
  \item Combining java classes and libraries as integrated quasi statically
  linked packages causes, that the resulting package already contains all
  functionally necessary code of the Library\footnote{$\rightarrow$ p.
  \pageref{RevEngDistributeStaticallyCombinedByteCode}}.
  \item Compiling java classes without combining them with the referred Library
  classes causes, that the compiled classes at least contain identifiers having
  been declared by the Library\footnote{$\rightarrow$ p.
  \pageref{RevEngDistributeDynamicallyLinkedCode}}.
  \item Combiling C/C++ files or classes and linking them with the referred
  Libaries statically causes, that the resulting executable indeed contains all
  functional relevant code of all used Libraries\footnote{$\rightarrow$ p.
  \pageref{RevEngDistributeStaticallyLinkedBinaries}}.
  \item Combiling C/C++ files or classes without linking them to the referred
  Libaries causes, that the resulting object file can dynamically be linked on
  another machine and contains identifiers offered by the Library and sometimes
  some functional code injected by dissolving some inline functions or
  macros\footnote{$\rightarrow$ p.
  \pageref{RevEngDistributeDynamicallyLinkedCode}}.
\end{itemize}

So -- overall -- the situation is this: The LGPL3-RevEng-Rule tells us that we
have to allow reverse engineering of the portions of the Library
contained in the Combined Work. The LGPL3 additionally specifies, that a Combined Work
is simply the result of technically combining the work using the Library (the
application) and the Library. Finally the praxis tells us, that (a) combining
both components statically indeed causes that the resulting Combined Work contains
portions of the Library\footnote{So, it is triggering the LGPL3-RevEng-Rule.},
and that (b) we -- in case of preparing the both parts as dynamically
combinable components -- still have to clarify whether the resulting work
already contains portions of the Library.

Just as the LGPL-v2, the LGPL-v3 supports us to answer this question by its §3
whose linguistic conjunctions we thoroughly have to consider:

\begin{quote}\emph{The object code form of an Application may incorporate
material from a header file that is part of the Library. \textbf{You may convey}
such object code under terms of your choice, \emph{provided that}, \textbf{[}
\textbf{if} the incorporated material is \textbf{not} limited to numerical
parameters, data structure layouts and accessors, or small macros, inline
functions and templates (ten or fewer lines in length) \textbf{]}, \textbf{you
do both} of the following: \textbf{a)} Give prominent notice with each copy of
the object code that the Library is used in it and that the Library and its use
are covered by this License. \textbf{b)} Accompany the object code with a copy
of the GNU GPL and this license document\textbf{]}\footcite[cf.][\nopage wp.,
§3; emphasis and additional braces KR.]{Lgpl30OsiLicense2007a}.}
\end{quote}

The first sentence of this paragraph tells us that he is dedicated to object
files which are compiled and not linked to the used Library, but which
nevertheless can contain portions of the Library. The second sentence regulates
the distribution of such object files and can be logically serialized:

\begin{alltt}
( \([\Lambda:]\)
  ( You \(\emph{compliantly distribute}\) object code [incorporating 
    material from the Library] under terms of your choice ) 
  \(\rightarrow\)  
  \([\Xi:]\)
  ( \([\omega:]\)
    ( the incorporated material is not limited to numerical
      parameters, data structure layouts and accessors, or 
      small macros, inline functions and templates 
      [ten or fewer lines in length] ) 
    \(\rightarrow\) 
    ( \([\alpha:]\) ( you do [a] \(\ldots]\) )
    \(\wedge\) \([\beta:]\) ( you do [b] \(\ldots]\) )
) ) )
\end{alltt}  

We see, that this LGPL3-sentence concerning the distribution of object files
contains a main rule (\emph{($\Lambda$ $\rightarrow$ $\Xi$)}) and that the
conclusion $\Xi$ itself has the form of an embedded sub rule (\emph{($\omega$
$\rightarrow$ ( $\alpha$ $\wedge$ $\beta$)}).

Firstly, the main rule enforces us to respect the sub rule if we want to
distribute the object code compliantly\footnote{follows by Modus Ponens to
\emph{($\Lambda$ $\rightarrow$ $\Xi$)}.}. Secondly, the main rule tells us that
we do not distribute the object code compliantly if we do not respect the sub
rule \footnote{follows by Modus Ponens to \emph{($\Lambda$ $\rightarrow$
$\Xi$)}.}.

We have two ways to respect the sub rule, and one way not to respect it:
\begin{itemize}
  \item If the object code contains more and/or larger elements of the Library
  than the limit specifies, then \textbf{we do respect the sub rule}, if we do
  $\alpha$ and $\beta$\footnote{follows by Modus Ponens to \emph{($\omega$
  $\rightarrow$ ($\alpha$ $\wedge$ $\beta$))}.}.
  \item If the object code contains elements of the Library at most up to
  specified limits, then \textbf{we do respect the sub rule} without having to
  do some additionally tasks\footnote{follows by definition of an implication:
  if the premise of this sub rule is false, the sub rule is as whole is true and
  hence respected.}
  \item But if the object code contains more and/or larger elements of the
  Library than the limit specifies \textbf{and} if we do not do $\alpha$
  \emph{or} $\beta$, then \textbf{we do not respect the sub
  rule}\footnote{follows from definition of an implication: if the premise is
  true and the conclusion is false, the the implication as whole is false, as
  well.}.
\end{itemize}

Thus, -- at the end and based on the additional object code specification and
the known empirical background knowledge concerning the software programming --
the LGPL3-RevEng-Rule delivers the same result as the
LGPL2-RevEng-Rule\footnote{$\rightarrow$
\pageref{RevEngLgpl2ComplianceByRenverseEngine}}:

\begin{itemize}  
  \item \emph{With respect to a LGPL-v3 licensed Library, you are not required
  to allow reverse engineering, if you [A] develop your work using the Library,
  on the base of a standard version of the Library containing the interfaces as
  the original developers have designed it, if you [B] compile your work using
  this Library, as a discrete (set of) dynamically linkable or combinable
  file(s), if you [C] use only the standard compilation methods which preserve
  the upstream approved interfaces\footnote{and which therefore do not to exceed
  the LGPL-v3 limits}, and if you [D] distribute the produced unlinked object
  code or bytecode files before they are linked as an executable.}
  \item \emph{In all other cases of distributing a work using such a Library,
  you are required to allow reverse engineering of the work using this Library
  -- especially, \ldots}
  \begin{itemize}
    \item \emph{if you distribute the work using the Library and the Library
    together as a statically linked program or as an integrated package
    containing both parts, the work using the library and the Library
    itself\footnote{This holds also if you distribute a script language based
    program or package, notwithstanding the fact, that one does not need the
    permission of reverse engineering to understand script language based
    applications}.}
    \item \emph{if you distribute a work containing manually copied portions of
    the Library.}
  \end{itemize}
\end{itemize}


%% use all entries of the bibliography
%\nocite{*}

