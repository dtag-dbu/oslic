% Telekom osCompendium 'for being included' snippet template
%
% (c) Karsten Reincke, Deutsche Telekom AG, Darmstadt 2011
%
% This LaTeX-File is licensed under the Creative Commons Attribution-ShareAlike
% 3.0 Germany License (http://creativecommons.org/licenses/by-sa/3.0/de/): Feel
% free 'to share (to copy, distribute and transmit)' or 'to remix (to adapt)'
% it, if you '... distribute the resulting work under the same or similar
% license to this one' and if you respect how 'you must attribute the work in
% the manner specified by the author ...':
%
% In an internet based reuse please link the reused parts to www.telekom.com and
% mention the original authors and Deutsche Telekom AG in a suitable manner. In
% a paper-like reuse please insert a short hint to www.telekom.com and to the
% original authors and Deutsche Telekom AG into your preface. For normal
% quotations please use the scientific standard to cite.
%
% [ Framework derived from 'mind your Scholar Research Framework' 
%   mycsrf (c) K. Reincke 2012 CC BY 3.0  http://mycsrf.fodina.de/ ]
%


%% use all entries of the bibliography
%\nocite{*}

\section{Excursion: What is a 'Derivative Work' - the basic idea of open source}
\footnotesize
\begin{quote}\itshape
This chapter briefly discusses existing attempts to define the derivated works of
technical aspects, like dynamical or statical linking or not. We will
prove that linking can not deliver a definite criteria: 1) modules are only
unzipped libraries. 2) you can distribute software as modules added by a script,
which statically(sic!) links all modules before executing the program. 3) The
criteria of pipe-communication is good, but not sufficient. 4) All these
attempts do not match the constituting features of script languages. Therefore we
will follow Moglen(?) and will argue from the viewpoint of a developer: it is
only a question of a function, method or anything else which calls (jumps into)
a piece of code which has been licensed by a license protecting
on-top-developments and you have a derivated work.
\end{quote}
\normalsize
\ldots

\textit{\emph{IMPORTANT}: This text is only a first draft which sketches some
aspects. Later on we will fully elaborate the complete argumentation.}

As first version, we present an outline of the arguing structure this chapter
will later part of a more narrative way.

\begin{description}
  \item[The meaning 'derivative work' must be known!] Many open source licenses
  are using the the term 'derivative work'  [cite the sources], either direct or
  indirect in form of the work 'modification' [Write a table as survey]. And
  nearly all licenses, which are using the term 'derivative work' etc., are
  linking tasks which must be executed to comply with the corresponding license,
  to the precondition, that something is a derivative work.
  [table survey]. \textbf{Hence, for acting in accordance to such a license, it
  has to be known, what a derivate work is}
  \item[Unfortunately the meaning is not clearly fixed]. The exist some
  different readings of the term 'derivative work' [specify the differences and
  cite the sources] \textbf{Hence, it is not as clear as wished what a derivative
  work is}
  \item[Hence lets argue from the viewpoint of a benevolent developer]: Open
  source licenses are written for software developers, mostly to preserve their
  freedom, to develop software. And sometimes these licenses are also written by
  software developers -- or at least by the assistance of. So, one should be
  able to answer the question under which circumstances a piece of software is a
  'derivative work' of another piece of software on the base of two principles:
  \begin{itemize}
  \item Let us argue on the base of a benevolent neutral software developer
  without hidden interests or a hidden agenda.
  \item In case of doubts let us preferably assume that the two pieces
  interrelate as source and derivative work -- so that the OSLiC rather recommends
  to execute the required tasks than to put them away.
\end{itemize}
\end{description}

Basically we generalize a specific viewpoint of the LGPL: It uses three terms:

\begin{description}
  \item[\enquote{library}] is defined as \enquote{a collection of software
  functions and/or data prepared so as to be conveniently linked with
  application programs}\footcite[cf.][\nopage wp §0]{Lgpl21OsiLicense1999a}.
  \item[\enquote{work based on the library}] is defined as \enquote{either the
  library or any derivative work}\footcite[cf.][\nopage wp
  §0]{Lgpl21OsiLicense1999a}.
  \item[\enquote{work that uses the library}] is defined as something which
  initially \enquote{[\ldots] is not a derivative work of the library [\ldots]}
  but can become a derivative work by being combined / linked to the library it
  uses\footcite[cf.][\nopage wp §5]{Lgpl21OsiLicense1999a}.
\end{description}

Following these specifications, one has to conclude that there can be derived
\emph{derivative works} of the library in two different ways: First, the library
itself can be enhanced without changing the character of being a library. Then,
of course, the resulting library is a derivative work of the initial library.
Second, an overaching program can use the library by calling functions, methods
or data, offered by the library. In this case, the overarching program
functionally depends on the library and is a derivative work (as soon as it is
linked to the library).

This viewpoint can be generalized: also snippets, modules, plugins can be
enhanced and used by overarching programs or even by more complex libraries.
Based on this viewpoint - which should finally be formulated as the viewpoint of
a benevolent impartial developer - the OSLiC uses the following rules by which
the OSLiC decides to take something as derivative Work:
\label{sec:BenevolentDerivativeWorkUnderstanding}

\begin{description}
  \item[Copy-Case] Copying a piece of code from a source file and pasting it
  into a target file makes the target file a derivatve work of the source
  file\footnote{ Be careful: this case must still be distinguished from the case
  of an automatically inclusion (header files, script libraries) during the
  compilation / execution: Header files allon should not evekoe a derivative
  work.}.
  \item[Modify-Case] Inserting any new content or deleting any existing content
  of a source file makes the resulting target file being a derivate work of the
  source file.
  \item[Call-Case] Inserting into a target file the call of function which is
  defined inside of and delivered by a sourcefile makes the target file
  depending on the source file and therefore a derivative work of the delivering
  source file.
\end{description}

And here are some applications of the rules

\begin{itemize}
  \item Combining the copy-case and the modify-case on the base of two
  different source files make the resulting target file being a derivative work
  of the two source files (follows from copy-case and modify-case)
  \item \ldots
\end{itemize}

[\ldots TBD \ldots]

%\bibliography{../../../bibfiles/oscResourcesEn}
