% Telekom osCompendium bib data check file English
%
% (c) Karsten Reincke, Deutsche Telekom AG, Darmstadt 2011
%
% This LaTeX-File is licensed under the Creative Commons Attribution-ShareAlike
% 3.0 Germany License (http://creativecommons.org/licenses/by-sa/3.0/de/): Feel
% free 'to share (to copy, distribute and transmit)' or 'to remix (to adapt)'
% it, if you '... distribute the resulting work under the same or similar
% license to this one' and if you respect how 'you must attribute the work in
% the manner specified by the author ...':
%
% In an internet based reuse please link the reused parts to www.telekom.com and
% mention the original authors and Deutsche Telekom AG in a suitable manner. In
% a paper-like reuse please insert a short hint to www.telekom.com and to the
% original authors and Deutsche Telekom AG into your preface. For normal
% quotations please use the scientific standard to cite.
%
% [ File structure derived from 'mind your Scholar Research Framework' 
%   mycsrf (c) K. Reincke CC BY 3.0  http://mycsrf.fodina.de/ ]

%
% select the document class
% S.26: [ 10pt|11pt|12pt onecolumn|twocolumn oneside|twoside notitlepage|titlepage final|draft
%         leqno fleqn openbib a4paper|a5paper|b5paper|letterpaper|legalpaper|executivepaper openrigth ]
% S.25: { article|report|book|letter ... }
%
% oder koma-skript S.10 + 16
\documentclass[DIV=calc,BCOR=5mm,11pt,headings=small,oneside,abstract=true, toc=bib]{scrartcl}

%%% (1) general configurations %%%
\usepackage[utf8]{inputenc}

%%% (2) language specific configurations %%%
\usepackage[]{a4,ngerman}
\usepackage[ngerman, german, english]{babel}
\selectlanguage{english}

%language specific quoting signs
%default for language emglish is american style of quotes
\usepackage{csquotes}

% jurabib configuration
\usepackage[see]{jurabib}
\bibliographystyle{jurabib}
\input{../btexmat/oscJbibCfgEnInc}

% language specific hyphenation
\input{../btexmat/oscHyphenationEnInc}

%%% (3) layout page configuration %%%

% select the visible parts of a page
% S.31: { plain|empty|headings|myheadings }
%\pagestyle{myheadings}
\pagestyle{headings}

% select the wished style of page-numbering
% S.32: { arabic,roman,Roman,alph,Alph }
\pagenumbering{arabic}
\setcounter{page}{1}

% select the wished distances using the general setlength order:
% S.34 { baselineskip| parskip | parindent }
% - general no indent for paragraphs
\setlength{\parindent}{0pt}
\setlength{\parskip}{1.2ex plus 0.2ex minus 0.2ex}


%%% (4) general package activation %%%
%\usepackage{utopia}
%\usepackage{courier}
%\usepackage{avant}
\usepackage[dvips]{epsfig}

% graphic
\usepackage{graphicx,color}
\usepackage{array}
\usepackage{shadow}
\usepackage{fancybox}

%- start(footnote-configuration)
%  flush the cite numbers out of the vertical line and let
%  the footnote text directly start in the left vertical line
\usepackage[marginal]{footmisc}
%- end(footnote-configuration)

\begin{document}

%% use all entries of the bliography
\nocite{*}

%%-- start(titlepage)
\titlehead{bibliography test}
\subject{For testing \itshape{new bibliographisc data}}
\title{With Commented Bibliography}
\subtitle{and text inside of the footnotes}
\author{K. Reincke\input{../btexmat/oscLicenseFootnoteInc}}
%\thanks{den Autoren von KOMA-Script und denen von Jurabib}
\maketitle
%%-- end(titlepage)

\begin{abstract}
\noindent \itshape
Bibliographic references must be tested carefully before they are really used.
That's the purpose of this testtext: A new bibliographic entry of the database
is tested as part of the most important use cases: the initial use of the work,
the directly following reuse - firstly as quote of the same page, then as quote
of another page of the same work -, followed by another work of the same author
and again the first quote which now show arise as abbreviated reference. If the
quoted work is part of a collection then the collection should be tested in
similar contexts. Finally the complete commented bibliography is printed.
\end{abstract}

\section{Initial quotation \& Repetition by $\backslash$footcite out of
\emph{jurabib}}

This line simulates the \enquote{ initial quotation which therefore should be
referred by the complete set of all bibliographic data}\footcite[cf.
besides:][S.123ff]{Buchtala2007a}, followed by a \enquote{ quotation out of the same work
and the same page}\footcite[cf.][S.123ff]{Buchtala2007a}, which again is
followed by a \enquote{  quotation out of the same work but a different
page.}\footcite[cf.][S.125f]{Buchtala2007a}.

If possible there should now be cited another work of the same
author\footcite[cf.][S.321]{Buchtala2007a}.

Now we cite a complete other work\footcite[cf.
additionally][S.42]{Spielkamp2008a} for being able to test wether the \enquote{
short-title-abbreviated-quotation'}\footcite[cf.
furthermode][S.123]{Buchtala2007a} is correctly used.

\section{Testzitat $\backslash$cite in $\backslash$footnote}

Now we check wether we can correctly use our data also by the command
$\backslash$cite as part of the command $\backslash$footnote - firstly the
``page differing quotation''\footnote{\cite[cf.][S.125]{Buchtala2007a}}, followed by
another work\footnote{\cite[cf.][S.42]{Spielkamp2008a}} for evoking the
`short-title-abbreviated-quotation''\footnote{\cite[cf.][S.125]{Buchtala2007a}}
again.


\small
\bibliography{../bibfiles/oscResourcesEn}

\end{document}
