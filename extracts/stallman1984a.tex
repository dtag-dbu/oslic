% Telekom osCompendium extract template
%
% (c) Karsten Reincke, Deutsche Telekom AG, Darmstadt 2011
%
% This LaTeX-File is licensed under the Creative Commons Attribution-ShareAlike
% 3.0 Germany License (http://creativecommons.org/licenses/by-sa/3.0/de/): Feel
% free 'to share (to copy, distribute and transmit)' or 'to remix (to adapt)'
% it, if you '... distribute the resulting work under the same or similar
% license to this one' and if you respect how 'you must attribute the work in
% the manner specified by the author ...':
%
% In an internet based reuse please link the reused parts to www.telekom.com and
% mention the original authors and Deutsche Telekom AG in a suitable manner. In
% a paper-like reuse please insert a short hint to www.telekom.com and to the
% original authors and Deutsche Telekom AG into your preface. For normal
% quotations please use the scientific standard to cite.
%
% [ File structure derived from 'mind your Scholar Research Framework' 
%   mycsrf (c) K. Reincke CC BY 3.0  http://mycsrf.fodina.de/ ]

%
% select the document class
% S.26: [ 10pt|11pt|12pt onecolumn|twocolumn oneside|twoside notitlepage|titlepage final|draft
%         leqno fleqn openbib a4paper|a5paper|b5paper|letterpaper|legalpaper|executivepaper openrigth ]
% S.25: { article|report|book|letter ... }
%
% oder koma-skript S.10 + 16
\documentclass[DIV=calc,BCOR=5mm,11pt,headings=small,oneside,abstract=true, toc=bib]{scrartcl}

%%% (1) general configurations %%%
\usepackage[utf8]{inputenc}

%%% (2) language specific configurations %%%
\usepackage[]{a4,ngerman}
\usepackage[ngerman, german, english]{babel}
\selectlanguage{english}

%language specific quoting signs
%default for language emglish is american style of quotes
\usepackage{csquotes}

% jurabib configuration
\usepackage[see]{jurabib}
\bibliographystyle{jurabib}
\input{../btexmat/oscJbibCfgEnInc}

% language specific hyphenation
\input{../btexmat/oscHyphenationEnInc}

%%% (3) layout page configuration %%%

% select the visible parts of a page
% S.31: { plain|empty|headings|myheadings }
%\pagestyle{myheadings}
\pagestyle{headings}

% select the wished style of page-numbering
% S.32: { arabic,roman,Roman,alph,Alph }
\pagenumbering{arabic}
\setcounter{page}{1}

% select the wished distances using the general setlength order:
% S.34 { baselineskip| parskip | parindent }
% - general no indent for paragraphs
\setlength{\parindent}{0pt}
\setlength{\parskip}{1.2ex plus 0.2ex minus 0.2ex}


%%% (4) general package activation %%%
%\usepackage{utopia}
%\usepackage{courier}
%\usepackage{avant}
\usepackage[dvips]{epsfig}

% graphic
\usepackage{graphicx,color}
\usepackage{array}
\usepackage{shadow}
\usepackage{fancybox}

%- start(footnote-configuration)
%  flush the cite numbers out of the vertical line and let
%  the footnote text directly start in the left vertical line
\usepackage[marginal]{footmisc}
%- end(footnote-configuration)

\begin{document}

%% use all entries of the bliography

%%-- start(titlepage)
\titlehead{Literaturexzerpt}
\subject{Autor(en): Stallman / Stallman1984a}
\title{Titel: The GNU Manifesto}
\subtitle{Jahr: 1984 / 2002 }
\author{K. Reincke\input{../btexmat/oscLicenseFootnoteInc}}

%\thanks{den Autoren von KOMA-Script und denen von Jurabib}
\maketitle
%%-- end(titlepage)
%\nocite{*}

\begin{abstract}
\noindent
\cite[(in:)][]{StaGay2002a} \\
\noindent
\cite[(ist:)][]{Stallman1984a} \\
Das Werk / The work\footcite[][]{Stallman1984a} \\
\noindent \itshape
\ldots Eines der frühestes Dokumente, die die Idea des GNU Projektes und der GNU
Software skizzieren. Der längste Teil des Artikels diskutiert die Vorteile der
freien Software für alle Beteiligten, selbst im Hinblick auf eine kommerzielle
Nutzung.
\\
\noindent
\ldots One of the earliest documents delineating the idea of the GNU project and
GNU software. The larger part of the article discusses the question that free
software offers advantages for all users, even in the context of a commercial
use.
\end{abstract}
\footnotesize
%\tableofcontents
\normalsize

\section{Line of Thought}
Following RMS the GNU Manifesto \enquote{[\ldots] was written at the
beginning of the GNU Project, to ask for participation and
support}\footcite[cf.][31]{Stallman1984a}. It contains

\begin{itemize}
  \item \ldots a short description of that software which already exists in
  1984\footcite[cf.][31]{Stallman1984a}
  \item \ldots the affirmation, that \enquote{GNU will be able to run Unix
  programs, but will not be identical to
  Unix}\footcite[cf.][31]{Stallman1984a}. RMS declares that Unix is
  \enquote{not (his) ideal system} but good enough that he \enquote{[\ldots]
  can fill in what Unix lacks without spoiling
  them}\footcite[cf.][32]{Stallman1984a}. [This remark hints to the
  modular structure of unix: many, but simple and small tools which just do what
  they are designed to do]
  \item \ldots the explanation, that \enquote{GNU is not in the public
  domain} and that \enquote{everyone will be permitted to modify and
  redistribute GNU, but no distributor will be allowed to restrict its
  further redistribution}\footcite[cf.][32]{Stallman1984a}. At this time
  there still doesn't exist any GPL and hence these sentences might be read as
  first incarnation of the idea of free software.
  \item \ldots a longer discussion of using / making free software in a
  commercial environment\footcite[cf.][34ff]{Stallman1984a}
\end{itemize}

\section{Specific Aspects}

In the beginning RMS uses a wording which he himself specifies as
\enquote{careless}\footcite[cf.][31, FN 1]{Stallman1984a}. This carelessness
concerns the meaning of 'free'. From the point of reviewing his older texts RMS
adds very often the hint, that 'free' is used in the sense of 'freedom', not in
the sense, \enquote{[\ldots] that copies of GNU should always be
distributed at little nor no charge}\footcite[cf.][31, FN
1]{Stallman1984a}. Therefore RMS also rejects his earlier statement, that
\enquote{once GNU is written, everyone will be able to obtain good system
software free, just like air}\footcite[cf.][34]{Stallman1984a}
\small
\bibliography{../bibfiles/oscResourcesEn}

\end{document}
