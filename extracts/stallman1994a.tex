% Telekom osCompendium extract template
%
% (c) Karsten Reincke, Deutsche Telekom AG, Darmstadt 2011
%
% This LaTeX-File is licensed under the Creative Commons Attribution-ShareAlike
% 3.0 Germany License (http://creativecommons.org/licenses/by-sa/3.0/de/): Feel
% free 'to share (to copy, distribute and transmit)' or 'to remix (to adapt)'
% it, if you '... distribute the resulting work under the same or similar
% license to this one' and if you respect how 'you must attribute the work in
% the manner specified by the author ...':
%
% In an internet based reuse please link the reused parts to www.telekom.com and
% mention the original authors and Deutsche Telekom AG in a suitable manner. In
% a paper-like reuse please insert a short hint to www.telekom.com and to the
% original authors and Deutsche Telekom AG into your preface. For normal
% quotations please use the scientific standard to cite.
%
% [ File structure derived from 'mind your Scholar Research Framework' 
%   mycsrf (c) K. Reincke CC BY 3.0  http://mycsrf.fodina.de/ ]

%
% select the document class
% S.26: [ 10pt|11pt|12pt onecolumn|twocolumn oneside|twoside notitlepage|titlepage final|draft
%         leqno fleqn openbib a4paper|a5paper|b5paper|letterpaper|legalpaper|executivepaper openrigth ]
% S.25: { article|report|book|letter ... }
%
% oder koma-skript S.10 + 16
\documentclass[DIV=calc,BCOR=5mm,11pt,headings=small,oneside,abstract=true, toc=bib]{scrartcl}

%%% (1) general configurations %%%
\usepackage[utf8]{inputenc}

%%% (2) language specific configurations %%%
\usepackage[]{a4,ngerman}
\usepackage[ngerman, german, english]{babel}
\selectlanguage{english}

%language specific quoting signs
%default for language emglish is american style of quotes
\usepackage{csquotes}

% jurabib configuration
\usepackage[see]{jurabib}
\bibliographystyle{jurabib}
\input{../btexmat/oscJbibCfgEnInc}

% language specific hyphenation
\input{../btexmat/oscHyphenationEnInc}

%%% (3) layout page configuration %%%

% select the visible parts of a page
% S.31: { plain|empty|headings|myheadings }
%\pagestyle{myheadings}
\pagestyle{headings}

% select the wished style of page-numbering
% S.32: { arabic,roman,Roman,alph,Alph }
\pagenumbering{arabic}
\setcounter{page}{1}

% select the wished distances using the general setlength order:
% S.34 { baselineskip| parskip | parindent }
% - general no indent for paragraphs
\setlength{\parindent}{0pt}
\setlength{\parskip}{1.2ex plus 0.2ex minus 0.2ex}


%%% (4) general package activation %%%
%\usepackage{utopia}
%\usepackage{courier}
%\usepackage{avant}
\usepackage[dvips]{epsfig}

% graphic
\usepackage{graphicx,color}
\usepackage{array}
\usepackage{shadow}
\usepackage{fancybox}

%- start(footnote-configuration)
%  flush the cite numbers out of the vertical line and let
%  the footnote text directly start in the left vertical line
\usepackage[marginal]{footmisc}
%- end(footnote-configuration)

\begin{document}

%% use all entries of the bliography

%%-- start(titlepage)
\titlehead{Literaturexzerpt}
\subject{Autor(en): Stallman / Stallman1994a}
\title{Titel: Why Software Should Not Have Owners}
\subtitle{Jahr: 1994 / 2002 }
\author{K. Reincke\input{../btexmat/oscLicenseFootnoteInc}}

%\thanks{den Autoren von KOMA-Script und denen von Jurabib}
\maketitle
%%-- end(titlepage)
%\nocite{*}

\begin{abstract}
\noindent
\cite[(in:)][]{StaGay2002a} \\
\noindent
\cite[(ist:)][]{Stallman1994a} \\
Das Werk / The work\footcite[][]{Stallman1994a} \\
\noindent \itshape
\ldots  Wiederholt den Standpunkt, dass die Gesellschaft Software benötige, die
sie lesen, berichtigen, anpassen und verbessern könne. Stattdessen würde Eigner
von Software 'Black Boxes' liefern, die man nicht untersuchen könne. Ferner
erwähnt der Artikel Möglichkeiten, wie man mit freier Software Geld verdienen
könne, was zudem als eine wichtige Möglichkeit herausgestellt werde.
\\
\noindent
\ldots This article repeats the statement that the society needs programs that
people can read, fix, adapt, and improve. Instead of this software owners
deliver black boxes that can't be studied. Additionally the article exemplifies
some important methods to earn money with free software .
\end{abstract}
\footnotesize
%\tableofcontents
\normalsize

\section{Line of Thought}

This article repeats the statement that the \enquote{society [\ldots] needs
[\ldots] programs that people can read, fix, adapt, and improve, not [SW] just
operate.}. But instead of this software owners \enquote{[\ldots]
typically deliver black boxes [\ldots] that we can't study or
change}\footcite[cf.][47f]{Stallman1994a}

 Additionally the article exemplifies some important methods to earn money with
 free software. It mentions (a) the FSF itself which \enquote{[\ldots] raises
 funds by selling GNU CD-ROMs, T-shirts, manuals, and deluxe
 distributions}, the company 'Cygnus Support' and (c) even the Air force,
 which powered the ADA compiler development\footcite[cf.][48]{Stallman1994a}
 
 

\section{Specific Aspects}

\small
\bibliography{../bibfiles/oscResourcesEn}

\end{document}
