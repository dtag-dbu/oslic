% Telekom osCompendium extract file
%
% (c) Karsten Reincke, Deutsche Telekom AG, Darmstadt 2011
%
% This LaTeX-File is licensed under the Creative Commons Attribution-ShareAlike
% 3.0 Germany License (http://creativecommons.org/licenses/by-sa/3.0/de/): Feel
% free 'to share (to copy, distribute and transmit)' or 'to remix (to adapt)'
% it, if you '... distribute the resulting work under the same or similar
% license to this one' and if you respect how 'you must attribute the work in
% the manner specified by the author ...':
%
% In an internet based reuse please link the reused parts to www.telekom.com and
% mention the original authors and Deutsche Telekom AG in a suitable manner. In
% a paper-like reuse please insert a short hint to www.telekom.com and to the
% original authors and Deutsche Telekom AG into your preface. For normal
% quotations please use the scientific standard to cite.
%
% [ File structure derived from 'mind your Scholar Research Framework' 
%   mycsrf (c) K. Reincke CC BY 3.0  http://mycsrf.fodina.de/ ]

%
% select the document class
% S.26: [ 10pt|11pt|12pt onecolumn|twocolumn oneside|twoside notitlepage|titlepage final|draft
%         leqno fleqn openbib a4paper|a5paper|b5paper|letterpaper|legalpaper|executivepaper openrigth ]
% S.25: { article|report|book|letter ... }

% oder koma-skript S.10 + 16
\documentclass[DIV=calc,BCOR=5mm,11pt,headings=small,oneside,abstract=true, toc=bib]{scrartcl}

%%% (1) general configurations %%%
\usepackage[utf8]{inputenc}

%%% (2) language specific configurations %%%
\usepackage[]{a4,ngerman}
\usepackage[english, german, ngerman]{babel}
\selectlanguage{ngerman}

%language specific quoting signs
%default for language emglish is american style of quotes
\usepackage{csquotes}

% jurabib configuration
\usepackage[see]{jurabib}
\bibliographystyle{jurabib}
\input{../btexmat/oscJbibCfgDeInc}

% language specific hyphenation
\input{../btexmat/oscHyphenationDeInc}

%%% (3) layout page configuration %%%

% select the visible parts of a page
% S.31: { plain|empty|headings|myheadings }
%\pagestyle{myheadings}
\pagestyle{headings}

% select the wished style of page-numbering
% S.32: { arabic,roman,Roman,alph,Alph }
\pagenumbering{arabic}
\setcounter{page}{1}

% select the wished distances using the general setlength order:
% S.34 { baselineskip| parskip | parindent }
% - general no indent for paragraphs
\setlength{\parindent}{0pt}
\setlength{\parskip}{1.2ex plus 0.2ex minus 0.2ex}


%%% (4) general package activation %%%
%\usepackage{utopia}
%\usepackage{courier}
%\usepackage{avant}
\usepackage[dvips]{epsfig}

% graphic
\usepackage{graphicx,color}
\usepackage{array}
\usepackage{shadow}
\usepackage{fancybox}

%- start(footnote-configuration)
%  flush the cite numbers out of the vertical line and let
%  the footnote text directly start in the left vertical line
\usepackage[marginal]{footmisc}
%- end(footnote-configuration)


%- end(short-title-citation-configuration)


\begin{document}

%% use all entries of the bliography

%%-- start(titlepage)
\titlehead{Literaturexzerpt}
\subject{Autor(en): Kreutzer}
\title{Titel: Software veröffentlichen}
\subtitle{Jahr: 2008 }
\author{K. Reincke\input{../btexmat/oscLicenseFootnoteInc}}
%\thanks{den Autoren von KOMA-Script und denen von Jurabib}
\maketitle
%%-- end(titlepage)
%\nocite{*}

\begin{abstract}
\noindent
\cite[(in:)][]{DjoGehGraKreSpi2008a} \\
\noindent
\cite[(ist:)][163 - 167]{Kreutzer2008b} \\
\noindent \itshape
[DE]: Erläutert das initiale Zusammenfallen von Urheberrecht und
Verwertungsbefugnis. Letzteres kann übertragen werden, auch generell per
Arbeitsrecht: angestellte Programmierer übergeben die Nutzungsrechte an ihren
Werken mit dem Arbeitsvertrag an ihren Arbeitgeber. Wichtig aber auch:
Urheberrecht impliziert keinen Ideenschutz. \\
\noindent
[EN]: Specifies that 'Urheberrecht' and 'Verwertungsbefugnis' are initially
linked in Germany. The right to use ('Verwertungsbefugnis') can be assigned to
other, even generally: In Germany employed developers assign their right to
determine the software use to their employers by signing the contract. But: the
German 'Urheberrecht' doesn't include a protection of the embedded ideas.
\end{abstract}
\footnotesize
%\tableofcontents
\normalsize

\section{Gedankengang}

Die Analyse geht davon aus, dass der Urheber als \enquote{Schöpfer des
Werkes} - bei Software also der Programmierer - \enquote{[\ldots] in der
Regel auch die exklusive Verwertungsbefugnis
(habe)}\footcite[vgl.][163]{Kreutzer2008b}:

\begin{quote}{\itshape \enquote{Er [der Urheber resp. der Programmierer (K.R.)]
kann darüber entscheiden, was er mit dem Programm macht, ob er es ausschließlich
selbst nutzen oder anderen die Nutzung gestatten will - und zu welchen
Bedingungen}\footcite[][163]{Kreutzer2008b}}\end{quote}

Bei echt kollaborierend erstellten Werken sei dies dann ein Recht, das
konsequenterweise nur gemeinsam ausgeübt werden könne\footcite[vgl.][164]{Kreutzer2008b}. 

Bei Programmierern im Angestelltenverhältnis werde das Nutzungs- resp.
Verwertungsrecht an der erstellten Software hingegen \enquote{schon nach dem
Gesetz} - also nicht nur durch den Arbeitsvertrag selbst - an den
Arbeitgeber übertragen: \enquote{Ein Lizenzvertrag [\ldots] (müsse) dafür nicht
geschlossen werden}, das Gehalt gelte bereits als Vergütung \enquote{der
Erwerb der Rechte}\footcite[vgl.][164]{Kreutzer2008b}.

Zu Markierung der Urheberrechte am Sourcecode - auch für den Fall, dass Rechte
durchgesetzt werden müssen - böte sich zunächst einmal der Copyright-Vermerk an;
allerdings sei dieser praktischen Grenzen unterworfen\footcite[vgl.][166f]{Kreutzer2008b}.

Wichtig zu wissen, sei ferner, \enquote{[\ldots] dass das Urheberrecht keinen
Ideenschutz (gewähre)}: Bestimmen resp.untersagen könne der \enquote{Urheber}
nur die Verwendung eines konkret gestalteten \enquote{Computerprogramms}, die
Verwendung einer mit dem Programm konkretisierten Idee in einem anderen
Konkretisierungsrahmen sei davon nicht
betroffen\footcite[vgl.][167]{Kreutzer2008b}.

\section{Allgemeine Anmerkungen}

Eine interessante Frage bei dem Copyright-Vermerk betrifft die Unterscheidung
zwischen Autor resp. Urheber und den Nutzungsrechten: 

Wenn ich es richtig
verstanden habe, bleibe ich der Autor eines Programmes selbst dann, wenn ich die
Nutzungsrechte abgetreten habe: Niemand kann also auftreten und sagen, dies oder
jenes stamme von ihm, wenn ich es geschaffen habe. Und das gilt selbst dann,
wenn dieser andere von mir die Nutzungsrechte erworben hat.

Im Falle einer Firma wäre also zu fragen, wer der Copyright-Inhaber ist: der
Autor oder sie, die Firma. Wer also muss genannt werden, die Firma oder der
Urheber?

\small
\bibliography{../bibfiles/oscResourcesDe}

\end{document}
